\documentclass{ximera}

\input{../../preamble.tex}
\author{Jim Talamo}
\title[Refresh:]{ L'H\^{o}pital's Rule}

\begin{document}
\begin{abstract}
Review  L'H\^{o}pital's Rule.
\end{abstract}
\maketitle

\begin{problem}
One of the most powerful tools for computing limits is L'H\^{o}pital's Rule.

\begin{theorem}
Suppose that $f$ and $g$ are differentiable on an open interval $I$ except possibly at a point $c$ contained in $I$ and $g'(x) \neq 0$
for all $x$ near $c$.  If either

\[ \lim _{x\to c}f(x)=0 \textrm{ and } \lim _{x\to c}g(x)=0 \quad  \textrm{ -OR- }  \quad \lim _{x\to c}f(x)= \pm \infty \textrm{ and } \lim _{x\to c}g(x)=\pm \infty\]

and $\lim_{x \to c} \frac{f'(x)}{g'(x)}$ exists, then

\[
\lim_{x \to c} \frac{f(x)}{g(x)} = \lim_{x \to c} \frac{f'(x)}{g'(x)}.
\]
\end{theorem}

The theorem above can be generalized to handle the case where $x \to \infty$ as well, and the results can be summarized by noting that when we have functions $f$ and $g$ that meet the criteria of the theorem

\begin{quote}
When a limit has the indeterminate form $\frac{0}{0}$ or $\pm \frac{\infty}{\infty}$, we can apply L'H\^{o}pital's Rule.
\end{quote}

\begin{example}
Compute $\lim_{x \to \infty} \frac{\ln(x)}{x^2}$.

\begin{explanation}
Notice that direct evaluation gives the indeterminate form $\frac{\infty}{\infty}$, so we can use L'H\^{o}pital's Rule to conclude

\[
\lim_{x \to \infty} \frac{\ln(x)}{x^2} = \lim_{x \to \infty} \frac{1/x}{2x} = \lim_{x \to \infty} \frac{1}{2x^2} =0
\]
\end{explanation}
\end{example}

Now, try using L'H\^{o}pital's Rule to compute the following limits.  If a limit does not exist, use $\infty$ or $-\infty$ as appropriate or write ``DNE'' otherwise.

\begin{exercise}
\[
\lim_{x \to \infty} \frac{\ln^3(x)}{x^{1/2}} = \answer{0}  
\]
\end{exercise}

\begin{exercise}
\[
\lim_{x \to \infty} \frac{e^{x/2}}{x^2+2x+3} = \answer{\infty}  
\]
\end{exercise}

\begin{exercise}
\[
\lim_{x \to 0} \frac{x \sin(x)}{e^x-1} = \answer{0}  
\]
\end{exercise}

\end{problem}
\end{document}
