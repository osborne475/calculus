\documentclass{ximera}

\input{../preamble.tex}
\author{Jim Talamo}

\outcome{Understand linear density and its connection to mass.}
\outcome{Calculate the mass of an objection with varying density.}
\outcome{Understand work and how it is computed.}
\outcome{Calculate work when force is varying.}
\outcome{Know when to integrate a cross-section to solve a physics problem.}
\outcome{Calculate work when distance is varying.}


\title[Dig-In:]{Physical applications}

\begin{document}
\begin{abstract}
We apply the procedure of ``Slice, Approximate, Integrate'' to model physical situations.
\end{abstract}
\maketitle

Thus far, we have studied several \emph{geometric} applications of the procedure  ``Slice, Approximate, Integrate.''  Indeed, it can be used to find lengths, areas, volumes.  This procedure is not limited to modeling only geometric situations.  Many problems from other STEM fields requires the same technique.  This section examines many of these situations.  First, we make a more generalized observation of the philosophy behind the ``Slice, Approximate, Integrate'' procedure.

\section{A broader perspective}
Let's take a step back and try to think about each of the situations we've used the ``Slice, Approximate, Integrate'' procedure to model.  For each quantity of interest - length, area, or volume- we used an object whose length, area, or volume we could calculate.  

\begin{itemize}
\item To find the area between curves, we know how to find the area of rectangles, so we approximate each slice by a rectangle.  
\item For solids of revolution, we can calculate the volume of washers or shell, so we approximate each slice by a washer or a shell. 
\item For length of curves, we can find the length of a line segment, so we approximate each slice by a line segment.
\end{itemize}

In physics, we take measurable quantities from the real world and attempt to find meaningful relationships between them.  These relationships are often expressed using formulas, but these come with assumptions or restrictions.  If these requirements are not met, we can still apply the ``Slice, Approximate, Integrate'' procedure to compute physical quantities of interest.

%A basic example of this would be the physical ideal of \dfn{force}. Force applied to an object changes the acceleration of an object:
%
%\[
%\mathrm{force} = \mathrm{mass} \cdot \mathrm{acceleration}.
%\]
%and while we can put a  physical interpretation to this arithmetical
%definition, at the end of the day force is simply ``mass times
%acceleration.'' The SI unit of force is a \dfn{newton}, which is
%defined to be
%\[
%1\unit{N} = 1\unit{kg}\cdot \unit{m}/\unit{s}^2. 
%\]
%
%\begin{warning}
%  The concepts of \dfn{mass} and \dfn{weight} are closely related, yet
%  different, concepts. The mass $m$ of an object is a quantitative
%  measure of that object's resistance to acceleration. The weight $w$
%  of an object is a measurement of the force applied to the object by
%  the acceleration of gravity $g$.
%\end{warning}
%
%\begin{question}
%  A cubic meter of water has a mass of $1000\unit{kg}$. Approximately,
%  what is its weight?
%  \begin{multipleChoice}
%    \choice{$1000\unit{kg}$}
%    \choice{$1000\unit{N}$}
%    \choice{$1000\unit{lbs}$}
%    \choice{$98000\unit{kg}$}
%    \choice[correct]{$98000\unit{N}$}
%    \choice{$98000\unit{lbs}$}
%  \end{multipleChoice}
%  \begin{feedback}
%    Weight is a unit of force.
%  \end{feedback}
%\end{question}
%
%
%\begin{question}
%  To get a feel for what a newton is, consider this: if an apple has a
%  mass of $0.1\unit{kg}$, what force would an apple exert on your hand
%  due to the acceleration due to gravity?
%  \begin{prompt}
%  \[
%  F = (0.1) \cdot (-9.8) = \answer[given]{-0.98}\unit{N},
%  \]
%  \end{prompt}
%\begin{feedback}
%  Hence the ``weight'' of an apple is approximately $1\unit{N}$.
%\end{feedback}
%\end{question}

\section{Mass of a wire with variable density}

Given a physical object, its density is a measure of how the mass of the object is distributed.  In the case where the density of the object is \emph{constant}, we have the following.

Three dimensions: $\left<\textrm{density}\right> = \frac{\left<\textrm{mass}\right>}{\left<\textrm{volume}\right>}$

Two dimensions: $\left<\textrm{density}\right> = \frac{\left<\textrm{mass}\right>}{\left<\textrm{area}\right>}$

One dimension: $\left<\textrm{density}\right> = \frac{\left<\textrm{mass}\right>}{\left<\textrm{length}\right>}$

We can (and often do) approximate physical objects, like wires, as one dimensional or thin sheets as two dimensional. 

\begin{question}
  If a wire has linear density $5 \unit{g}/\unit{m}$, how many grams
  will $4 \unit{m}$ of this wire be?
  \begin{prompt}
  \[
  \mathrm{Mass} = \answer[given]{20} \unit{g}
  \]
  \end{prompt}
  \begin{hint}
    In this case, we need just multiply the density by the length.
  \end{hint}
\end{question}

\begin{fact}
The kilogram ($kg$) and meter ($m$) are the standard SI units of mass and length.  The standard unit for linear density is thus $kg/m$.
\end{fact}

Sometimes the linear density of an object can vary from one part of
the object to another. In this case, density will be a non-constant function, which we will represent by the Greek letter, \textit{rho}, $\rho$ (pronounced ``rho'').  In this case, the above formula no longer allows us to compute the mass of the wire.  Let's tackle this scenario with a motivating example.

\begin{model}
 Suppose that a wire that extends from $x=0$ to $x=2$ has density profile $\rho(x) = \frac{1}{(1+2x)^2}$.  How do we calculate its mass?

\begin{multipleChoice}
\choice{The length of the wire is 3, so the mass is $\rho(3) \cdot 3 = \frac{7}{49}$ units.}
\choice[correct]{We cannot use the result $\left<\textrm{density}\right> = \frac{\left<\textrm{mass}\right>}{\left<\textrm{length}\right>}$ because the density is not constant!}
\end{multipleChoice}

What should we do?  Let's try the ``Slice, Approximate, Integrate'' procedure.

\paragraph{Step 1: Slice} We cut the wire into numerous pieces.  One such piece of width $\Delta x$ is shown.

\begin{image}
\begin{tikzpicture}

\begin{axis}
	[
	domain=0:1.2, ymax=.9,xmax=1.2, ymin=-.9, xmin=-.2,
	axis lines=center, 
	xlabel=$x$,
	xtick={4},
	ylabel=$y$,
	ytick={4},
	every axis y label/.style={at=(current axis.above origin),anchor=south},
	every axis x label/.style={at=(current axis.right of origin),anchor=west},
	axis on top,
	typeset ticklabels with strut,
	]

	\addplot [draw=penColor,thick, smooth,domain=0:1] {.1}; 
	\addplot [draw=penColor,thick, smooth,domain=0:1] {-.1};	
	\draw[penColor,thick] (axis cs:1,-.1) -- (axis cs:1,.1);
	\draw[penColor,thick] (axis cs:.8,-.1) -- (axis cs:.8,.1);
	\draw[penColor,thick] (axis cs:.68,-.1) -- (axis cs:.68,.1);
	\draw[penColor,thick] (axis cs:.5,-.1) -- (axis cs:.5,.1);
	\draw[penColor,thick] (axis cs:.3,-.1) -- (axis cs:.3,.1);
	\draw[penColor,thick] (axis cs:.15,-.1) -- (axis cs:.15,.1);
	
	
	
	\addplot [name path=C,domain=0:1,draw=none] {-.1};   
	\addplot [name path=D,domain=0:1,draw=none] {.1};
	\addplot [fillp] fill between[of=C and D];
	\addplot [name path=A,domain=.68:.8,draw=none] {-.1};   
	\addplot [name path=B,domain=.68:.8,draw=none] {.1};
	\addplot [penColor!50] fill between[of=A and B];

	\draw[|-|,black,thick] (axis cs:.68,-.15) -- (axis cs:.8,-.15);
	\node at (axis cs:.74,-.25) [black] {$\Delta x$};
	
\end{axis}

\end{tikzpicture}
\end{image}

\paragraph{Step 2: Approximate} We know how to calculate the mass of a one dimensional object is when the density is constant.  This is a rather poor approximation for the entire rod, but if each slice is very small, it is a fairly good approximation.  For example, consider the slice that extends from $x=1$ to $x=1.001$.  Using the formula $\rho(x)  = \frac{1}{(1+2x)^2}$, we find to $5$ decimal places: $\rho(1) = \answer[given,tolerance=.0005]{.11111}$ and $\rho(1.001) = \answer[given,tolerance=.0005]{.11096}$.  In this case, the density doesn't vary too much over a small slice, so treating the density as a constant along the slice is a good approximation.

Since we know how to calculate the mass of an object with constant density, we thus approximate that

\begin{multipleChoice}
\choice[correct]{the density along each slice is constant, but each slice may have a different density than the others}
\choice{the density along each slice is constant, and each slice has the same density as all of the other slices}
\choice{the density of the whole rod is constant.}
\end{multipleChoice}

Since the density along is slice is constant, we find that the mass $\Delta m$ of a single slice is given by 

\[
\Delta m = \rho(x^*) \Delta x.
\]
where $x^*$ is an $x$-value along the slice where the density is approximated.

Let $x^*_k$ denote the $x$-value used to determine the density of the $k$-th slice, and $\Delta x_k$ be the width of the $k$-slice.  The total approximate mass $m$ of the rod is thus

\[
m = \sum_{k=0}^n \rho(x_k^*) \Delta x_k.
\]

\paragraph{Step 3: Integrate}  The usual procedure converts this approximate mass into the exact mass; indeed the definite integral will perform the simultaneous limiting process of shrinking the width of each slice while adding up the contributions from all of them.  Since the mass of a single slice was found to be $\Delta m = \rho \Delta x$, the total exact mass is found the same way as before.

\[
m = \int_{x=0}^{x=2} \rho(x) \d x
\]

Now we can compute the mass of the wire.

\begin{align*}
m &= \int_{x=0}^{x=2} \frac{1}{(1+2x)^2} \d x \\
&= \eval{\answer[given]{-\frac{1}{2}(1+2x)^{-1}} }_0^2 \\
&= \answer[given]{\frac{2}{5}}
\end{align*}
\end{model}

There was nothing particularly special about this example.  In fact, we can summarize the results in a formula.

\begin{formula}
Suppose a thin wire lies on the $x$-axis between $x=a$ and $x=b$ and has a piecewise continuous density function given by $\rho(x)$ on $x=a$ to $x=b$.  Then, the mass of the wire from $x=a$ to $x=b$ is given by

\[
m = \int_{x=a}^{x=b} \rho(x) \d x.
\]
\end{formula}

\begin{remark}
Some texts will refer to the density function $\rho(x)$ as the \emph{density profile} of the wire.
\end{remark}

\begin{example}
A thin wire that extends from $x=0$ to $x=4$ has density given by

\[ \rho(x) = \left\{ 
\begin{array}{cl}
4x, & 0 \leq x\leq 2 \\
8 ,& 2 < x \leq 4.
\end{array} \right.
\]

Without performing any calculations, which half of the rod has more mass?

\begin{multipleChoice}
\choice{the half from $x=0$ to $x=2$}
\choice[correct]{the half from $x=2$ to $x=4$}
\choice{They both have equal mass.}
\end{multipleChoice}

\begin{explanation}
Note that the density of the left half of the rod is always less than 8 since $4x \leq 8$ for $0\leq x \leq 2$.  Thus, the left half should have less mass than the right side.


We now find the total mass of the rod.  It is still true that

\[
m = \int_{x=0}^{x=4} \rho(x) \d x.
\]
but since $\rho$ is now a piecewise function, we will need to compute the integral in pieces.  

\begin{align*}
m = \int_{x=0}^{x=4} \rho \d x & = \int_{x=0}^{x=2} \answer[given]{4x} \d x + \int_{x=\answer[given]{2}}^{x=\answer[given]{4}} \answer[given]{8} \d x \\
&= \eval{\answer[given]{2x^2}}_0^2+\eval{\answer[given]{8x}}_2^4 \\
&= \answer[given]{24} \textrm{ units}
\end{align*}
\end{explanation}

Let's now find the exact value of $a$ so that the mass of the wire from $x=0$ to $x=a$ is exactly half of the total mass of the wire. 

\begin{explanation}
Since we have reasoned that the right half of the rod has more mass than the left half, we expect that

\begin{multipleChoice}
\choice{$a<2$.}
\choice{$a=2$.}
\choice[correct]{$a>2$.}
\end{multipleChoice}
 
 Since we expect that $a>2$, we can find $a$ by setting:
 
 \[
 \int_{x=a}^{x=4} \rho \d x = \frac{1}{2} \int_{x=0}^{x=4} \rho \d x. \\
 \]
 
 Since $a>2$, we can use the bottom piece for $\rho$ in the first integral, and since we found that the total mass of the wire is $24$ units, we have the following.
 
 
 \begin{align*}
  \int_{x=a}^{x=4} 8 \d x &= \frac{1}{2} \cdot 24
  \eval{8x}_a^4 &=12 \\
  \answer[given]{32-8a} = 12 \\
  a &= \answer[given]{\frac{5}{2}}
  \end{align*}
  
  \begin{remark}
  This value of $a$ is called the \dfn{center of mass} of the wire.  
  
Note that we could have found it by setting $\int_{x=0}^{x=a} \rho \d x$ equal to 12, but this would have required two integrals since we expected that $a>2$.  Nevertheless, its not bad practice to show that the part of the rod from $x=0$ to $x=a=5/2$ is exactly half the mass of the rod:
  
  \[
  \int_{x=0}^{x=5/2} \rho \d x =   \int_{x=0}^{x=2} 4x \d x +  \int_{x=2}^{x=5/2} \rho \d x =\eval{2x^2}_0^2+\eval{8x}_0^{5/2} = 12
  \]

  \end{remark}
   \end{explanation}

\end{example}


\section{Work}
One of the most important concepts in physics is that of \dfn{work}, which measures the change in energy that occurs when a force moves an object over a certain displacement.  For those familiar with physics, the Work-Energy Theorem is a powerful tool for studying situations in Newtonian mechanics (such as a box sliding down an incline plane). 

For a \emph{constant} force $F$ acting on a particle over a displacement $d$ \emph{in the direction of displacement}, work is given by the formula:

\[ \left<\textrm{Work} \right>=  \left<\textrm{Force} \right> \cdot \left<\textrm{displacement} \right> \]

Denoting work by $W$, force by $F$, and displacement by $d$, we can write:

\[ W = F \cdot d. \]

\begin{fact}
The Newton ($N$) and Joule ($J$) are the standard SI units of force and work/energy.  The Newton is defined to be the amount of force required to give a mass of 1 $kg$ and acceleration of $1m/s^2$.  The Joule is then defined to be the amount of work done by a force of 1 $N$ while displacing  an object 1 $m$ in the direction of the force.
\end{fact}

\begin{remark}
While some background knowledge with physics certainly gives important context to these examples, it is not necessary to perform the required mathematics.  In fact, we can think of the formula $W=F\cdot d$ as ``When the quantity $F$ is constant over a displacement $d$,  $W$ is some quantity calculated by the formula $W=F \cdot d$''.
\end{remark}

\begin{remark}
Note that the formula $W=F \cdot d$ comes with the requirement that the force be \emph{constant} over the motion!  There are many examples of important forces in physics that are not constant:

\begin{itemize}
\item The force required to stretch a spring $x$ units from its equilibrium position is given by $F(x) = k x$, where $k$ is a constant of proportionality.
\item The \emph{attractive gravitational} force between two objects of masses $M$ and $m$ that are separated by a distance of $r$ units  is $F(r) = \frac{GMm}{r^2}$, where $G$ is a constant.  
\item The \emph{electrostatic} force between two particles with charges $Q$ and $q$ that are separated by a distance of $r$ units  is $F(r) = \frac{kQq}{r^2}$, where $k$ is a constant.
\end{itemize}
\end{remark}
%On the other hand, \dfn{work} is defined to be accumulated force
%\textit{in the direction of motion} over a distance.
%\begin{question}
%  Which of the following are examples where work of this kind is being done?
%  \begin{selectAll}
%    \choice{studying calculus}
%    \choice[correct]{a car applying breaks to come to a stop over a distance of $100\unit{ft}$}
%    \choice[correct]{a young mathematician climbing a mountain}
%    \choice{a young mathematician standing still, holding a $1000$ page calculus book for $10$ minutes}
%    \choice{a young mathematician walking around with a $1000$ page calculus book}
%    \choice[correct]{a young mathematician picking up a $1000$ page calculus book}
%  \end{selectAll}
%  \begin{feedback}
%    While studying calculus may ``feel'' like work, it is not
%    (typically) an example of an accumulated force over a distance,
%    and hence no work is done.
%
%    On the other hand, a car applying breaks is a change in motion, and
%    hence a force is applied. Since this force is applied over a
%    distance, work is done.
%
%    Climbing a mountain is also an example of work, as one is applying
%    force to overcome the acceleration due to gravity, over the
%    distance that one is climbing.
%
%    No work is done when holding a calculus book, as there is no
%    accumulated force over a distance.
%
%    It is also the case that no work is done when one walks around
%    with a calculus book, this is because the ``force'' is in a
%    direction perpendicular to the motion.
%
%    Finally, when one picks up a calculus book, you are moving the
%    book against the force due to the acceleration due to
%    gravity. Hence work is done.
%  \end{feedback}
%\end{question}
%
%When a \textbf{constant} force $F$ is applied to move an object a
%distance $d$, the amount of work performed is $W=F\cdot d$.  We can
%write the definition of work in the language of calculus as,
%\[
%W = \int_{a}^{b} F(x) \d x.
%\]
%The SI unit of work is a \dfn{joule}. To help understand this, $1$
%joule is approximately how much work is done when you raise an apple
%one meter.
%
%Let's again see why this is true.
%\begin{example}
%  If an apple has a mass of $0.1\unit{kg}$, how much work is required
%  to lift this apple $1$ meter?  Assume that the acceleration due to
%  gravity is $-9.8\unit{m}/\unit{s}^2$.
%  \begin{explanation}
%    Well, work is computed by
%    \[
%    W = \int_{a}^{b} F(x) \d x.
%    \]
%    Since force is mass times acceleration,
%    \begin{align*}
%      F(x) &= 0.1\cdot \answer[given]{(-9.8)} \\
%      &= \answer[given]{-0.98}.
%    \end{align*}
%    So, our integral becomes
%    \begin{align*}
%      \int_{0}^{1} \answer[given]{-0.98} \d x &= \eval{\answer[given]{-0.98 x}}_0^1\\
%      &=\answer[given]{-0.98}.
%    \end{align*}
%    Ah! So when lifting an apple $1$ meter, requires $\answer[given]{-0.98}$ joules of
%    work. The sign is negative since we are lifting \textbf{against}
%    the gravitational force.
%  \end{explanation}
%\end{example}
%In imperial units (as often used in the United States), force is
%measured in pounds ($\unit{lb}$) and distance is measured in feet
%($\unit{ft}$, hence work is measured in foot-pounds.
%

In what follows, we will look at two scenarios in which the formula $W=F \cdot d$ does not immediately apply.  They result when either the force required to move an object is not constant, or when different parts of the object we move must be moved different distances.  They can be summarized by:

%\begin{itemize}
%  \item \textbf{accumulating finite forces over infinitesimal distances}
%  \item \textbf{accumulating infinitesimal forces over large distances}.
%\end{itemize}
%Let's see an example of each type of situation:

\subsection{Work done by a non-constant force on a particle}

What will happen if the work done over the displacement is not constant?  We study another motivating example that leads to a more general result.

\begin{model} Suppose that a non-constant force $F(x)$ acts on an object from $x=a$ to $x=b$.  To find how much work does the force do on the object, what should we do?  Let's try the ``Slice, Approximate, Integrate'' procedure.

\paragraph{Step 1: Slice} We divide the displacement into numerous pieces.  One such piece of width $\Delta x$ is shown.

\begin{image}
\begin{tikzpicture}

\begin{axis}
	[
	domain=0:1.2, ymax=.9,xmax=1.2, ymin=-.9, xmin=-.2,
	axis lines=center, 
	xlabel=$x$,
	xtick={4},
	ylabel=$y$,
	ytick={4},
	every axis y label/.style={at=(current axis.above origin),anchor=south},
	every axis x label/.style={at=(current axis.right of origin),anchor=west},
	axis on top,
	typeset ticklabels with strut,
	]

	
	\draw[penColor,thick] (axis cs:.8,-.05) -- (axis cs:.8,.05);
	\draw[penColor,thick] (axis cs:.68,-.05) -- (axis cs:.68,.05);
	\draw[penColor,thick] (axis cs:.5,-.05) -- (axis cs:.5,.05);
	\draw[penColor,thick] (axis cs:.3,-.05) -- (axis cs:.3,.05);
	\draw[penColor,thick] (axis cs:.15,-.05) -- (axis cs:.15,.05);
	\draw[penColor,thick] (axis cs:1,-.05) -- (axis cs:1,.05);

	\addplot [draw=penColor,ultra thick, smooth,domain=.15:1] {0}; 
	
	\draw[|-|,black,thick] (axis cs:.68,-.15) -- (axis cs:.8,-.15);
	\node at (axis cs:.74,-.25) [black] {$\Delta x$};
	\node at (axis cs:.15,-.1) [black] {$a$};
	\node at (axis cs:1,-.1) [black] {$b$};
	
\end{axis}

\end{tikzpicture}
\end{image}

\paragraph{Step 2: Approximate} The only instance in which we know how to calculate the work is when force is constant.  If the force is continuous and each slice is very small, it is a good approximation.  Indeed, since we know how to calculate $W$ when $F$ is constant by the formula $W=F\cdot d$, we approximate that the force along each piece is constant, 

\begin{multipleChoice}
\choice[correct]{but the force along neighboring pieces may be different.}
\choice{and the force along all neighboring pieces must be the same as the force on the indicated one.}
\choice{because the force along the entire displacement is constant.}
\end{multipleChoice}

Since the force along is slice is constant, we find that the work $\Delta W$ done by the force along to displace the object $\Delta x$ along the single slice is given by 

\[
\Delta W = F(x^*) \Delta x.
\]
where $x^*$ is the $x$-value along the slice where the force is approximated.

Since there are $n$ slices used, let $x^*_k$ denote the $x$-value used to determine the force along the $k$-th slice, and $\Delta x_k$ be the width of the $k$-slice.  The total approximate work $W$ done by the force is thus

\[
W = \sum_{k=0}^n F(x_k^*) \Delta x_k.
\]

\paragraph{Step 3: Integrate}  The usual procedure converts this approximate work into the exact work; indeed the definite integral will perform the simultaneous limiting process of shrinking the width of each slice while adding up the contributions from all of them.  Indeed, since the work done by the force along a single slice was found to be $\Delta W =F \Delta x$, the total exact force is found by

\[
W = \int_{x=a}^{x=b} F(x)  \d x.
\]

\end{model}


%%%%

%The basic set-up for these integrals is as follows:
%  \begin{image}[2in]
%  \begin{tikzpicture}[scale=2,every node/.style={transform shape}]
%    \node at (0,0) {
%      $\mathrm{Work} = \int_a^b \underbrace{\mathrm{(force)}\d x}_{\d W}$
%    };
%  \end{tikzpicture}
%\end{image}
Let's try some examples to see the formula in action.

%%%Good example, but requires a better explanation that I do not have time to give right now-MAKE INTO EXERCISE%%%%%%%
%\begin{example}
%  How much work is performed pulling a $60\unit{m}$ climbing rope up a
%  cliff face, where the rope has a mass of $66\unit{g}/\unit{m}$?
%  \begin{explanation}
%    Let $y$ be the distance we have pulled the rope.  The force $F(y)$
%    on the rope is always ``large'' (read: not infinitesimal), but the
%    force is changing as we pull the rope up.  If we think of pulling
%    the rope up only an infinitesimal amount $\d y$, then the force
%    will be constant over that infinitesimal distance, so the
%    infinitesimal amount of work done by moving the rope through $\d
%    y$ at height $y$ is
%    \[
%    \d W = F(y) \d y.
%    \]
%    Integrating these infinitesimal amounts of work from $y=0$ to
%    $y=60$ should yield the total amount of work done.  In this
%    example, the mass of the rope remaining after $y$ feet have been
%    pulled up is
%    \[
%    \mathrm{mass}(y) = \answer[given]{(0.066)(60-y)} \unit{kg}.
%    \]
%    \begin{hint}
%      There are $60-y$ feet of rope, at a density of $66\unit{g}/\unit{m}$, so
%      the total mass of the rope is $(60-y)66\unit{g}$.  Converted to kilograms,
%      this is $(60-y)(0.066)\unit{kg}$.
%    \end{hint}
%    The force due to gravity on the rope after $y$ feet have been pulled up is
%    \[
%    F(y) = \answer[given]{(0.066)(60-y)(9.8)} \unit{N} 
%    \]
%    \begin{hint}
%      To get the force due to gravity from the mass, we just
%      multiply the mass by standard gravity, $g = 9.8
%      \frac{\unit{m}}{\unit{s}^2}$.  Since we have already written
%      our mass in $\unit{kg}$, we obtain $F(y) = (0.066)(60-y)(9.8)
%      \unit{N}$.
%    \end{hint}
%    The total work done is
%    \[
%    \int_0^{60} F(y) \d y = \answer[given]{1164.24} \unit{J}
%    \]
%    \begin{hint}
%      \begin{align*}
%	\int_0^{60} F(y) \d y &= \int_0^{60}(0.066)(60-y)(9.8) \d y \\
%	&= (0.066)(9.8) \eval{\answer[given]{60y - \frac{y^2}{2}}}_0^{60}\\
%	&=(0.066)(9.8)(1800)\\
%	&= 1164.24
%      \end{align*}
%    \end{hint}    
%    By comparison, consider the work done in lifting the entire rope
%    $60$ meters. The rope weights
%    \[
%    60\times 0.066 \times 9.8 = 38.808\unit{N},
%    \]
%    so the work applying this force for $60$ meters is
%    \[
%    60\times 38.808 = 2328.48 \unit{J}.
%    \]
%    This is exactly twice the work calculated before, we leave it to
%    the young mathematician to understand why.
%  \end{explanation}
%\end{example}

\begin{example}
A common example of a variable force is the force required to stretch a spring.

\begin{formula}[Hooke's Law]\index{Hooke's Law}
  The force required to compress or stretch a spring $x$ units from
  its natural length (its unstretched length) is proportional to $x$.
  That is, the force is given by:
  \[
  F(x) = k\cdot x
  \]
  for some constant $k$, known as the \dfn{spring constant}.
\end{formula}

\begin{fact}
The SI units for the spring constant is $N/m$ (force/distance).  
\end{fact}

\begin{warning}
Units are important.  Make sure that before starting a problem, all quantities are expressed in terms of the same units.  If they are not, you will have to convert them!
\end{warning}

\begin{question}
  If a a force of $1\unit{N}$ stretches a given spring $2\unit{cm}$,
  then how far will a force of $5\unit{N}$ stretch the spring?
  \begin{prompt}
    The spring will stretch $\answer[given]{10}\unit{cm}$.
  \end{prompt}
  \begin{question}
    What is the spring constant in $N/m$ in this case?
    \begin{hint}
      Convert the distances to meters.
    \end{hint}
    \begin{prompt}
      $k= \answer[given]{50}\unit{N}/\unit{m}$.
    \end{prompt}
  \end{question}
\end{question}
\end{example}
  
  %Add back in once units are important
%\begin{example}
%  Say a force of $20\unit{lb}$ stretches a spring from its natural
%  length of $7$ inches to a length of $12$ inches. How much work was
%  performed in stretching the spring to this length?
%  \begin{explanation}
%    \begin{hint}
%      This is a ``Large forces over infinitesimal distances'' problem.
%    \end{hint}
%    Let $x$ be the amount we have stretched the spring beyond its
%    natural length in inches, and let $F(x)$ be the force exerted by
%    the spring at this distance in pounds.  We have that $F(0)=0$,
%    $F(5) = 20$, and we know that it is linear in the distance by
%    Hooke's law.  So the ``large force'' involved is $F(x) = 4x$.  The
%    work done by moving the spring from $x$ to $x+\d x$ (an
%    infinitesimal distance) is
%    \[
%    \d W = \answer[given]{4x} \d x.
%    \]
%    We need to accumulate these ``large forces over infinitesimal
%    distances'' from $x=0$ to $x=5$.
%    \begin{align*}
%      \mathrm{Work} &= \int_0^5 4x \d x\\
%      &= \eval{\answer[given]{2x^2}}_0^5\\
%      &=50
%    \end{align*}
%  \begin{hint}
%    We need the answer to be in foot-pounds not inch-pounds, so we convert
%    inches to feet to obtain $\frac{50}{12}$ foot-pounds.
%  \end{hint}
%    \[
%    \mathrm{Work} = \answer[given]{\frac{50}{12}} \unit{ft}\cdot\unit{lb}.
%    \]
%  \end{explanation}
%\end{example}

\begin{example}
Suppose that $40 N$ of force is required to stretch a spring $4m$ from its equilibrium position.  How much work is done by the force?

\begin{explanation}
Since the force is not constant, we want to use

\[W=\int_{x=a}^{x=b} F(x) \d x = \int_0^4 kx \d x . \]

We do not yet know the value of the spring constant, but since $40 N$ is required to displace the spring $4m$, we find by using Hooke's Law.

\begin{align*}
F &= kx \\
40 &= k(4) \\
k &= \answer[given]{10} N/m
\end{align*}

Thus, 

\begin{align*}
W= \int_0^4 kx \d x &= \int_0^4 10x \d x \\
&= \eval{\answer[given]{5x^2}}_0^4 \\
&= \answer[given]{80} J
\end{align*}


\end{explanation}

\end{example}

%\begin{example}
%ROCKET SHIP - Also Write correspeonding exercise with fuel burning
%\end{example}



\subsection{Work done by a constant force on a collection of particles}

%The basic set-up for these integrals is as follows:
%  \begin{image}
%  \begin{tikzpicture}[scale=2,every node/.style={transform shape}]
%    \node at (0,0) {
%      $\mathrm{Work} = \int_a^b \underbrace{
%        \mathrm{(distance)}
%        \underbrace{
%          \mathrm{(acceleration)}
%          \underbrace{
%            \mathrm{(density)}
%            \underbrace{
%              \mathrm{(area)}\d x}_{\d V}}_{\d M}}_{\d F}}_{\d W}$
%    };
%  \end{tikzpicture}
%  \end{image}
%  Where $V$ represents volume, $M$ represents mass, $F$ represents
%  force, and $W$ represents work as a function of $x$, position.
%  Let's try some examples:
  
In the last problems, a non-constant force acted on a single particle.  Another scenario occurs when a constant force acts on a collection of particles.  The following example explores this situation in detail.

\begin{model}
  A cylindrical storage tank with a radius of $10 \unit{m}$ and a height of $30\unit{m}$ is filled with water, which has density
$\rho=1000 \unit{kg}/\unit{m}^3$. Compute the amount of work performed by pumping the water up to a point $5$ meters above the top of the tank.  You may use the fact that $g=9.8m/s^2$ and that the force exerted by gravity on an object of mass $m$ is given by $F=mg$.
  \begin{explanation}
  
%  \begin{warning}
%  Before continuing, there is one important idea to note.  \dfn{Weight} is defined as the force exerted on a body by gravity, that is:
%  \[
%  \mathrm{weight} = \mathrm{mass}\cdot\textrm{acceleration from gravity}
%  \]
%  This means that
%  \begin{quote}
%    \textbf{Weight is proportional to mass}.
%  \end{quote}
%  Since the two measurements are proportional, they are omen used  interchangeably in everyday conversation. However, when computing \textit{work} in certain contexts, we have to be careful to note which is given. When mass is given, it must be multiplied by the acceleration of  gravity to reference the related force.  We will approximate   standard gravity, on Earth, as $9.8m/s^2$ or   as $32m/s^2$.
%\end{warning}

Note that we need to overcome the force exerted by gravity to lim the liquid, which in this case is a constant $62.4 \unit{kg}/\unit{m}^3$.

\begin{multipleChoice}
\choice{Find the weight of the liquid of the tank.  Then, use $W=F \cdot d$, where $F$ will be the weight of the liquid in the tank and $d$ will be $5$ meters.}
\choice[correct]{We cannot use $W=F \cdot d$ because different parts of the water have to be moved different distances.}
\end{multipleChoice}

Note that if we were lifting the entire tank $35$ meters, there would be no need for calculus since each particle of water would be moved the same distance.  Here, different parts of the water must be moved different distances,  Indeed,  the water at the top of the tank has a much shorter distance to travel than the water at the bottom of the tank.
	
\paragraph{Step 1: Slice}  A key observation is that each particle of water at the same height must be moved the same distance, so we will slice the tank into vertical pieces.  Setting $y=0$ at the base of the tank, and examine a piece at height $y$.

\begin{image}
	\begin{tikzpicture}
	
	\fill [fillp] (0,0) ellipse (0.4 and 0.1);
	\fill [fillp] (-0.4,0) -- (-0.4,-1) arc (180:360:0.4 and 0.1) -- (0.4,0) arc (0:180:0.4 and 0.1);
	
	\draw (0,0) ellipse (0.4 and 0.1);
	\draw (-0.4,0) -- (-0.4,-1);
	\draw (-0.4,-1) arc (180:360:0.4 and 0.1);
	\draw [dashed,opacity=0.75] (-0.4,-1) arc (180:360:0.4 and -0.1);
	\draw (0.4,-1) -- (0.4,0);  
	
	\draw[decoration={brace,raise=.3cm},decorate,thin] (-0.4,-1) -- (-0.4,0);
	\node[anchor=east] at (-0.7,-0.5) {\tiny $30$m};
	
	\draw[decoration={brace,raise=.05cm},decorate,thin] (-0.4,-1) -- (-0.4,-.55);
	\node[anchor=east] at (-0.42,-0.77) {\tiny $y$};
	
	\draw[decoration={brace,raise=.1cm},decorate,thin] (0.4,-1.1) -- (0,-1.1);
	\draw[thin] (0.4,-1) -- (0,-1);
	\node[anchor=north] at (.2,-1.2) {\tiny $10$m};
	
	\node[anchor=north] at (.6,-.35) {\tiny $\Delta y$};
	
	\draw[thin,dashed] (-.8,.4) -- (.8,.4);
	\node[anchor=north] at (1.4,.6) {\tiny $h=35$ m};
		
	\draw[decoration={brace,raise=.1cm},decorate,thin] (0.4,-1.1) -- (0,-1.1);
	
	%slice
	%shade slice please!
	\draw (0,-.5) ellipse (0.4 and 0.1);
	\fill [penColor,opacity=0.5] (0,-.5) ellipse (0.4 and 0.1);
	\draw (-0.4,-.6) arc (180:360:0.4 and 0.1) -- (0.4,0) arc (0:180:0.4 and 0.1);
	\end{tikzpicture}
\end{image}

\paragraph{Step 2: Approximate} We know how to compute the work done by a constant force if we displace an object a certain distance.  Here, the force is constant, but each particle in the slice must be moved a different distance.  If the slice is thin enough, however, we can approximate that each particle within it must be moved the same distance.  That is, we can treat the slice as a single object and find the work $\Delta W$ required to move it using $\Delta W = \Delta F \cdot d$

Note here that the slice is at a height $y$ and needs to be limed to a height of $35$, so $d=\answer[given]{35-y}$.  

The amount of force $\Delta F$ is small because the slice width $\Delta y$ is small.  We must express $\Delta F$ in terms of $y$ since we sliced with respect to $y$.  First, note that gravity exerts a force of $\Delta F = (\Delta m) g $ on the slice.  Since the density of the liquid is constant, we have $\Delta m = \rho \Delta V$.  Note here that the cross-sections of the tank are circular, so $\Delta V = A \Delta y$.  We note that the cross-sectional area is constant no matter at what height $y$ the slice is drawn, so $A = \pi r^2 = 100 \pi$.

Putting it all together, we find

\[ \Delta W = \Delta F \cdot d = (\Delta m) g (35-y) = (\Delta V) \rho g (35-y) = A (\Delta y) \rho g (35-y) . \]

Rearranging this gives the approximate work required to move a single slice of liquid.

\[
\Delta W = \rho g A (35-y) \Delta y = 98000 \pi (35-y) \Delta y
\]

\paragraph{Step 3: Integrate} 

We can now find the exact work in the usual way.  We need to add up the contributions of all of the slices of liquid, which begin at $y=\answer[given]{0}$ and extend to $y=\answer[given]{30}$.  Thus, the exact work is given by

\[ W = \int_{y=0}^{y= \answer[given]{30}} 98000\pi(35-y) \d y. \]

Evaluating this integral is easier if we factor out the $98000\pi$ first.

\begin{align*}
W = \int_{y=0}^{y= 30} 98000\pi(35-y) \d y &= 98000\pi  \int_{y=0}^{y= 30} (35-y) \d y\\
&= 98000\pi \eval{35y-\frac{1}{2}y^2}_0^{30} \\
&= \answer[tolerance=100,given]{ 5.88 \times 10^7 \pi} J
\end{align*}
\end{explanation}

\end{model}

As usual, the process here can be generalized into another formula.  

\begin{formula}
Suppose that a tank extends from $y=0$ to $y=b$ and is filled with a liquid of density $\rho$.  If the cross-sectional area of the tank is given by a continuous function $A(y)$, then the work required to lift the liquid in the tank between $y=a$ and $y=b$ up to a height $y=h$ is given by

\[
W = \int_{y=a}^{y=b} \rho g A(y) D(y) \d y,
\]
where $g$ is the acceleration due to gravity and $D(y)$ is the distance that all particles at a given height $y$ must be moved.
\end{formula}

\begin{question}
A common instance in these problems requires that the liquid be moved above the vertical distance to which the tank is filled.  By using the conventions above, the mathematical condition for this is expressed by requiring that \wordChoice{\choice{$h<a$}\choice{$h>a$}\choice{$h<b$}\choice[correct]{$h>b$}}, and with this condition, $D(y)=$ \wordChoice{\choice{$b-h$}\choice{$h-b$}\choice{$b-y$}\choice{$y-b$}\choice[correct]{$h-y$}\choice{$y-h$}}.
\end{question}
%We will approximate that each 
%    Our approach will be to look at infinitesimal slabs of water.  All
%    of the water in such a slab will have to travel the same distance
%    to the top of the tank.  The force due to gravity on this slab
%    will be infinitesimal because the volume of the water is
%    infinitesimal.
%    
%    Let $y$ be the position of the slab as measured from the bottom of
%    the cylindrical tank and $\d y$ be the thickness of the slab.
%
%    \begin{image}
%      \begin{tikzpicture}
%        \draw[penColor,very thick] (0,2) ellipse (2 and .7);
%        \draw[very thick,penColor!20!background] (2,-2) arc (0:180:2 and .7);% top half of ellipse
%        \draw[very thick,penColor] (-2,-2) arc (180:360:2 and .7);% bottom half of ellipse
%      
%        \draw[very thick,penColor!20!background] (2,0) arc (0:180:2 and .7);% top half of ellipse
%        \draw[very thick,penColor] (-2,0) arc (180:360:2 and .7);% bottom half of ellipse
%        \draw[very thick,penColor!20!background] (2,0.1) arc (0:180:2 and .7);% top half of ellipse
%        \draw[very thick,penColor] (-2,0.1) arc (180:360:2 and .7);% bottom half of ellipse
%        
%        \draw[decoration={brace,mirror, raise=.1cm},decorate,thin] (-2,0)--(-2,-2);
%        
%        \node [penColor] at (-2.4,-1) {$y$};
%        \node [above,penColor] at (2.2,-0.3) {$\d y$};
%        
%        \draw[penColor, very thick] (2,2) -- (2,-2);
%        \draw[penColor, very thick] (-2,2) -- (-2,-2);
%      \end{tikzpicture}
%    \end{image}
%    Each slab has a volume of
%    \[
%    \pi r^2 \d y = \answer[given]{100}\pi\d y.
%    \]
%    Thus the weight of each slab is
%    \[
%    \answer[given]{62.4}(100\pi)\d y = \answer[given]{6240}\pi \d y.
%    \]
%    Note that this is already a force, not a mass measurement, so we
%    do not have to convert it further.  Each slab has to move a
%    distance of $\answer[given]{30-y +5}$, so the work done by moving
%    each slab is
%    \[
%    \d W = 6240 \pi (\answer[given]{35-y}) \d y.
%    \]
%    Thus the total work done by pumping all of the water to the top is
%    \begin{align*}
%      \mathrm{Work} &= \int_0^{30} 6240 \pi (35-y) \d y\\
%      &= 6240 \pi \eval{\answer[given]{35y-\frac{1}{2}y^2}}_0^{30}\\
%      &=  3744000 \pi \unit{m}\cdot\unit{kgs}.
%    \end{align*}
%  \end{explanation}
%\end{example}
\section{Final thoughts}
The technique of ``Slice, Approximate, Integrate'' can be used to solve physical problems as well as geometric ones.  These are not the only examples of physical problems that can be modeled and solved by using this technique.  The exercises will explore other problems, and you will run into many more in other STEM courses.

\begin{quote}
``Mathematics is the abstract key which turns the lock of the physical universe'' - John Polkinghome
\end{quote}

\end{document}
