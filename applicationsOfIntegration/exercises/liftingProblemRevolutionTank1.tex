\documentclass{ximera}

\input{../../preamble.tex}

\author{Jim Talamo and Alex Beckwith}
\license{Creative Commons 3.0 By-NC}


\outcome{Set up an integral that gives the length of a curve segment and evaluate it}

\begin{document}
\begin{exercise}

	A parabolic tank is formed by revolving the parabolic segment $y=x^2$ from $x=0$ to $x=3$ about the $y$-axis ($x$ is in meters). Suppose the tank is filled to a height $h$ with water ($\rho$=1000 kg/m$^3$). 
\begin{image}
\begin{tikzpicture}
\begin{axis}[
domain=-3:3,
xmin=-3.5, xmax=4.5,
ymin=-1, ymax=11,
ytick={12},
axis lines =center,
xlabel=$x$, ylabel=$y$, every axis y label/.style={at=(current axis.above origin),anchor=south},
every axis x label/.style={at=(current axis.right of origin),anchor=west},
axis on top,
]

\draw[penColor,very thick,smooth,fill=fill4] (axis cs: 0,9) ellipse (300 and 12);
\draw[penColor,very thick,smooth,fill=blue,opacity=0.25] (axis cs: 0,4) ellipse (200 and 8);
\draw[penColor,very thick,smooth] (axis cs:2,4) arc (360:180:200 and 8);
\draw[penColor,very thick,dashed] (axis cs:2,4) arc (0:180:200 and 8);
\addplot [penColor,very thick,smooth]	{x^2};

\addplot [name path=A,domain=-3:3,draw=none] {x^2};   
\draw [name path=B,draw=none] (axis cs: 3,9) arc (0:180:300 and 12);
\addplot [fill4,opacity=0.5] fill between[of=A and B];

\addplot [name path=C,domain=-2:2,draw=none] {x^2};   
\draw [name path=D,draw=none] (axis cs: 2,4) arc (0:180:200 and 8);
\addplot [blue,opacity=0.25] fill between[of=C and D];

\draw[decoration={brace,raise=.1cm},decorate,thin] (axis cs:2,4) -- (axis cs:2,0);
\node[anchor=west] at (axis cs:2.2,2) {$h$};

\addplot[thick, penColor2] plot coordinates {(0,9) (3,9)};
\addplot[thick, penColor2,dashed] plot coordinates {(3,0) (3,9)};
\node[anchor=west, penColor2] at (axis cs:3.2,4.5) {$x=3$};
         
\end{axis}
\end{tikzpicture}
\end{image}

The work required to move the water out of the top of the tank is given by the integral:

\[
W=\int_{y=\answer{0}}^{y=\answer{h}} \answer{9800\pi y (9-y)} \d y  
\]
(type your answer in terms of $h$ and  $\pi$)

\begin{hint}
The cross-sectional area is $A(y) = \pi r^2$.  Here $r$ is the distance from the axis of rotation to the slice at $(x,y)$ and this is a horizontal distance.  Use the equation that describes the curve to write $r$ in terms of $y$. 
\end{hint}

\begin{exercise}
The absolute value of the work required to move the water out of the bottom of the tank is given by the integral:

\[
W=\int_{y=\answer{0}}^{y=\answer{h}} \answer{9800\pi y^2} \d y  
\]
(type your answer in terms of $h$ and  $\pi$)

\begin{hint}
The slice at height $y$ now must be moved to $y=0$.  Thus, the distance $d(y) = \answer{y}$.
\end{hint}

\begin{exercise}

Determine the height $h>0$ so that the work required to pump the water out of the top of the tank is equal to the absolute value of the work required to pump the water out of the bottom of the tank.

\[
h=\answer{\frac{27}{4}} \text{m}
\]

\begin{hint}
Note that after setting the previous two results equal, the $9800 \pi$ can be factored out of the integrals on both sides of the equation.
\end{hint}

\end{exercise}
\end{exercise}
\end{exercise}
\end{document}
