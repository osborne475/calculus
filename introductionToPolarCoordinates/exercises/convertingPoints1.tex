\documentclass{ximera}

\input{../../preamble.tex}

%\outcome{Find tangent lines to parametric curves}
\author{Jim Talamo and Jason Miller}

\begin{document}
\begin{exercise}

Express the following polar coordinates in Cartesian coordinates: 

If $\left(r,\theta\right) = \left( 6, \frac{3\pi}{2}\right)$, then in Cartesian coordinates $(x,y) = \left( \answer{0}, \answer{-6 }\right)$.

If $\left(r,\theta\right) = \left(3, \frac{7 \pi}{4}\right)$, then in Cartesian coordinates $(x,y) = \left( \answer{\frac{3\sqrt{2}}{2}}, \answer{\frac{-3\sqrt{2}}{2}} \right)$

\begin{hint}
Use the relationships $x=r\cos(\theta)$ and $y=r\sin(\theta)$.
\end{hint}

\end{exercise}

\begin{exercise}

Express the following Cartesian coordinates in polar coordinates.  In your answer, use $r>0$ and $0 \leq \theta < 2\pi$. 


If $(x,y) =\left(2, -2 \right)$, then in polar coordinates $(r,\theta) =\left( \answer{ 2\sqrt{2}}, \answer{ \frac{7\pi}{4 } } \right)$. 

If $(x,y) =\left(-4, -4\sqrt{3} \right)$, then in polar coordinates $(r,\theta) =\left( \answer{ 8 } , \answer{  \frac{4\pi}{3}  } \right)$. 

\begin{hint}
Use the relationships $r^2=x^2+y^2$ and $\theta = \tan^{-1}\left(\frac{y}{x}\right)$.  Note that for the angle, there are generally two options; you can determine which one to use by examining the quadrant in which the point $(x,y)$ lies.

\end{hint}
\end{exercise}
\end{document}
