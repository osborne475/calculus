\documentclass{ximera}

\input{../../preamble.tex}

\author{Jim Talamo}
\license{Creative Commons 3.0 By-bC}


\outcome{}


\begin{document}
\begin{exercise}
Consider$\{a_n \}_{n=1}$ and let $s_n = \sum_{k=1}^{n} a_k$.  Suppose it is known that

\[
s_n = \frac{2n+1}{3n}
\]

Then, the sequence $\{s_n\}$ is:

\begin{selectAll}
\choice{increasing}
\choice[correct]{decreasing}
\choice[correct]{monotonic}
\choice[correct]{bounded above}
\choice[correct]{bounded below}
\choice[correct]{bounded}
\end{selectAll}
(Select all that apply)

\begin{hint}
We can write:

\[
s_n = \frac{2n+1}{3n} = \frac{2n}{3n}+\frac{1}{3n} = \answer{\frac{2}{3}}+\frac{1}{\answer{3n}}
\]
Hence, $s_n$ is:

\begin{multipleChoice}
\choice{increasing}
\choice[correct]{decreasing}
\end{multipleChoice}

\begin{question}
Since $\{s_n\}$ is decreasing, it must be:
\begin{multipleChoice}
\choice{bounded below}
\choice[correct]{bounded above}
\end{multipleChoice}

\begin{question}
We see $\lim_{n \to \infty} s_n = \answer{\frac{2}{3}}$.

If a sequence has a limit, then it:
\begin{multipleChoice}
\choice[correct]{must have a limit.}
\choice{could have a limit but does not have to have a limit.}
\choice{must not have a limit.}
\end{multipleChoice}

\end{question}
\end{question}

\end{hint}

The sequence $\{a_n \}_{n=2}$ is:
\begin{selectAll}
\choice{increasing}  %Definately not correct... writie out the first few terms
\choice{decreasing}
\choice[correct]{monotonic}
\choice[correct]{bounded above}
\choice[correct]{bounded below}
\choice[correct]{bounded}
\end{selectAll}
(Select all that apply)

\begin{hint}
Since $s_1=a_1$, we find: $a_1 = \answer{1}$.

Note, that after we specify the first term, we always have a recursive formula for $s_n$:

\[
s_n = s_{n-1}+a_n
\]
(we have to require $s_1=a_1$ in order to start this recursive formula!)

We can use the formula to find  and solve for $a_n$ for $n \geq 2$:

\[
a_n = \answer{\frac{1}{3n}-\frac{1}{3n-3}}
\]
Writing this as a simplified single rational function:

\[
a_n = \frac{\answer{-3}}{9n^2-3n}
\]

\end{hint}
\end{exercise}

\end{document}
