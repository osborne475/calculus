\documentclass{ximera}

\input{../../preamble.tex}

\author{Jim Talamo}
\license{Creative Commons 3.0 By-NC}


\outcome{Understand why a constant of integration is not needed when integrating $\d v$.}


\begin{document}
\begin{exercise}
This exercise explores why it is not necessary to include a constant when we integrate $\d v$. in the context of a specific example.

Consider the indefinite integral:
\[
\int x \sec^2(x) \d x 
\]

Suppose that we decide to include a constant of integration in the integration by parts procedure when we integrate $\d v$.

Let $u=\answer{x}$ so $\d u =\answer{1} \d x$ and $\d v = \sec^2(x) \d x$.  If we include a constant of integration, we find $v = \tan(x)+C$.

Thus, using the integration by parts formula:

\[
\int x \sec^2(x) \d x = x \left( \tan(x) +C \right) - \int \answer{\tan(x) +C} \d x
\]
(Use $C$ for the constant)

\begin{exercise}

Evaluating the integral and simplifying gives:

\[
\int x \sec^2(x) \d x = x \tan(x) +Cx - \left(\answer{\ln|\sec(x)| + Cx}\right) %there should be an additional "+C" term in the answer... but I don't know how you want to handle the two different arbitary constants.  Thanks!
\]
As you can see the terms with $C$ cancel!  It's easy to generalize this argument to ensure that is not necessary to include a constant of integration when integrating $\d v$.

\end{exercise}
\end{exercise}
\end{document}
