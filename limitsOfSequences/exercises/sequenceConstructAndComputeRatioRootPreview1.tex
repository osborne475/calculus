\documentclass{ximera}

\input{../../preamble.tex}

\author{Jim Talamo}
\license{Creative Commons 3.0 By-bC}


\outcome{}


\begin{document}
\begin{exercise}

Given a sequence $\{a_n\}_{n=1}$, there are two limits that can be constructed from it that play an important role later on.  This exercise gives practice constructing and computing them.

Consider the sequence $\{a_n \}_{n=1}$, where $a_n =n^2+2n-3$.  Then:
\[
\lim_{n \to \infty} \frac{a_{n+1}}{a_n} = \answer{1}
\]

\begin{hint}
The limit that must be computed is:

\[
\lim_{n \to \infty} \frac{a_{n+1}}{a_n} = \lim_{n \to \infty} \frac{\answer{(n+1)^2+2(n+1)-3}}{\answer{n^2+2n-3}}
\]

\end{hint}

\[
\lim_{n \to \infty} \sqrt[n]{a_n} = \answer{1}
\]
\begin{hint}
The limit that must be computed is:

\[
\lim_{n \to \infty} \sqrt[n]{a_n} = \lim_{n \to \infty} \sqrt[n]{\answer{n^2+2n-3}} =  \lim_{n \to \infty} \left(\answer{n^2+2n-3}\right)^{1/n}
\]
Set $L = \lim_{n \to \infty} \sqrt[n]{a_n}$, take the natural logarithm of both sides and use the properties of logarithms to bring the limit into a form where L'Hopital's rule applies.
\end{hint}
\end{exercise}
\end{document}
