\documentclass{ximera}

\input{../../preamble.tex}

\author{Jim Talamo}
\license{Creative Commons 3.0 By-bC}


\outcome{}


\begin{document}
\begin{exercise}

Select all of the following series that converge.  

\begin{selectAll}
\choice[correct]{$\sum_{k=1}^{\infty} \frac{k^2+2^k}{3^k}$}
\choice{$\sum_{k=3}^{\infty} \frac{4k^2+3k+1}{k^3+3k+2}$}
\choice[correct]{$\sum_{k=1}^{\infty} \frac{(-1)^k}{2k+3}$}
\choice{$\sum_{k=2}^{\infty} \frac{1}{\sqrt[3]{k}}$}
\choice[correct]{$\sum_{k=1}^{\infty} \frac{\sin(k)}{k^2}$}
\end{selectAll}

\begin{hint}
We say that a convergence test is \emph{applicable} if the assumptions of the test are met.  We say that a test is \emph{conclusive} if it can be used to determine whether a series converges or diverges.

%%%%%%%%%%%  FOR FIRST SERIES %%%%%%%%%%%
\begin{question}
For the series $\sum_{k=1}^{\infty} \frac{k^2+2^k}{3^k}$, is the series geometric or a $p$-series?

\begin{multipleChoice}
\choice{The series is a geometric series.}
\choice{The series is a $p$-series.}
\choice[correct]{The series is neither a geometric series nor a $p$-series.}
\end{multipleChoice}

 Which of the following tests are \emph{applicable}?

\begin{selectAll}
\choice[correct]{divergence test}
\choice[correct]{comparison test}
\choice[correct]{limit comparison test}
\choice[correct]{ratio test}
\choice[correct]{root test}
\choice{alternating series test}
\end{selectAll}

Since the summand contains a term that grows exponentially:

\begin{multipleChoice}
\choice[correct]{The ratio test and root test will likely be conclusive.}
\choice{The ratio test and root test will not be conclusive.} 
\end{multipleChoice}

Since it is easy to apply the growth rates results to find the dominant terms in the numerator and denominator:

\begin{multipleChoice}
\choice[correct]{The limit comparison will be conclusive.}
\choice{The limit comparison will not be conclusive.} 
\end{multipleChoice}

The comparison test requires that we know beforehand whether the series should converge or diverge and that we then choose to make an estimate accordingly.  Since we have three other options, it's better to try to use one of them instead! 
\end{question}

%%%%%%%%%%%  FOR SECOND SERIES %%%%%%%%%%%


\begin{question}
For the series $\sum_{k=3}^{\infty} \frac{4k^2+3k+1}{k^3+3k+2}$, is the series geometric or a $p$-series?

\begin{multipleChoice}
\choice{The series is a geometric series.}
\choice{The series is a $p$-series.}
\choice[correct]{The series is neither a geometric series nor a $p$-series.}
\end{multipleChoice}

Let's think about the other most useful convergence tests.  Which of the following tests are \emph{applicable}?

\begin{selectAll}
\choice[correct]{divergence test}
\choice[correct]{comparison test}
\choice[correct]{limit comparison test}
\choice[correct]{ratio test}
\choice[correct]{root test}
\choice{alternating series test}
\end{selectAll}

Since the summand $  \frac{4k^2+3k+1}{k^3+3k+2}$ is a rational function in $k$, which test is the best option?
\begin{selectAll}
\choice{divergence test}
\choice{comparison test}
\choice[correct]{limit comparison test}
\choice{ratio test}
\choice{root test}
\end{selectAll}
Note that since there is not a term that grows at least exponentially, the limits required in the ratio and root test will be $\answer{1}$ (no computations should be required!).  So, those tests will be inconclusive!


\end{question}

%%%%%%%%%%%  FOR THIRD SERIES %%%%%%%%%%%
\begin{question}
For the series $\sum_{k=1}^{\infty} \frac{(-1)^k}{2k+3}$, note that:

\begin{multipleChoice}
\choice[correct]{the series is alternating.}
\choice{the series is not alternating.}
\end{multipleChoice}

With alternating series, we can check for absolute convergence or try using the alternating series test.  Since the assumptions of the alternating series test are easy to verify in this case, use it!

Note also that since the positive portion of the summand is $\frac{1}{2k+3}$ and we expect that this diverges (it would be compared to $\frac{1}{2k}$ which is a $p$-series with $p=1$, it would not be a good idea to check for absolute convergence.

\end{question}

%%%%%%%%%%%  FOR FOURTH SERIES %%%%%%%%%%%
\begin{question}
For the series $\sum_{k=2}^{\infty} \frac{1}{\sqrt[3]{k}}$, is the series geometric or a $p$-series?

\begin{multipleChoice}
\choice{The series is a geometric series.}
\choice[correct]{The series is a $p$-series.}
\choice{The series is neither a geometric series nor a $p$-series.}
\end{multipleChoice}

Here, $p= \answer{\frac{1}{3}}$. 
\end{question}

%%%%%%%%%%%  FOR FIFTH SERIES %%%%%%%%%%%
\begin{question}

For the series $\sum_{k=1}^{\infty} \frac{\sin(k)}{k^2}$, is the series geometric or a $p$-series?

\begin{multipleChoice}
\choice{The series is a geometric series.}
\choice{The series is a $p$-series.}
\choice[correct]{The series is neither a geometric series nor a $p$-series.}
\end{multipleChoice}

Let's think about the other most useful convergence tests.  Which of the following tests are \emph{applicable}?

\begin{selectAll}
\choice[correct]{divergence test}
\choice{comparison test}
\choice{limit comparison test}
\choice{ratio test}
\choice{root test}
\choice{alternating series test}
\end{selectAll}

Unfortunately, the divergence test is not conclusive since $\lim_{n \to \infty} \frac{\sin(n)}{n^2} = \answer{0}$.  

Note that $\sin(k)$ is bounded, so in essence the summand should ``look like" $\frac{const}{k^2}$.  To establish this:

\begin{multipleChoice}
\choice[correct]{Check for absolute convergence}
\choice{use limit comparison test}
\end{multipleChoice} 

Thus, we will check whether the series:

\[
\sum_{k=1}^{\infty} \frac{|\sin(k)|}{k^2}
\]

converges.  Now that the summand is positive, we can use more tests!  Since we established that $\sin(k)$ is bounded, the test to use is:
\begin{multipleChoice}
\choice{ratio test}
\choice{root test}
\choice[correct]{comparison test}
\choice{limit comparison test}
\end{multipleChoice}

Note that since $|\sin(k)|<1$ for all $k$, we have $\frac{|\sin(k)|}{k^2}<\frac{1}{k^2}$ for all $k$.  The series $\sum_{k=1}^{\infty} \frac{1}{k^2}$:

\begin{multipleChoice}
\choice{is a geometric series with $|r|<1$.  It converges.}
\choice{is a geometric series with $|r|\geq1$.  It diverges.}
\choice[correct]{is a $p$-series with $p>1$.  It converges.}
\choice{is a $p$-series with $p \leq 1$.  It diverges.}
\end{multipleChoice}

Hence, by the comparison test:

\begin{multipleChoice}
\choice[correct]{$\sum_{k=1}^{\infty} \frac{|\sin(k)|}{k^2}$ converges.}
\choice{$\sum_{k=1}^{\infty} \frac{|\sin(k)|}{k^2}$ diverges.}
\end{multipleChoice}

So, the original series $\sum_{k=1}^{\infty} \frac{\sin(k)}{k^2}$:
\begin{multipleChoice}
\choice[correct]{converges absolutely and thus converges.}
\choice{converges absolutely but might diverge.}
\end{multipleChoice}


\end{question}
\end{hint}







\begin{quote}
You should be able to give a detailed solution for each of your choices.  Such a solution should include:

\begin{itemize}
\item What test you chose.
\item Why you are allowed to use that test.
\item The computations required for the conclusions of the test.
\item An explanation why the series either converges or diverges by the test you chose.
\end{itemize}

\end{quote}


\end{exercise}
\end{document}
